Several attempts to fix usernames and passwords have been made, for example, password managers make it much more feasible to use highly random passwords on a per-site basis, and second factor systems mitigate the risk of password loss. However, none of these systems solve all of the problems. Crucially, they make website login less convenient, which directly impacts user adoption. SQRL is an authentication system that aims to fix all of these problems. Whilst providing websites with no secrets to keep, and no way to link user accounts across website domains, the entire system is more convenient and far more secure. For example, a website, say amazon.com, presents the user with a QR code that changes every time the page is refreshed. To login, the user pulls out their smartphone, opens a SQRL app and scans the QR code, and that's it, they're logged in. Users without smartphones can also click the QR code, providing they have a SQRL client installed on their machine.

At it's core, SQRL is very simple, each user is given a unique master identity, a 256 bit randomly generated key. This key is used to HMAC the domain name of the website, resulting in another 256 bit value, used as the private key. The corresponding public key can be synthesised from this value and serves as the users identity. Finally, the cryptographic challenge contained by the QR code is signed using the private key in order to authenticate the user. The HMAC of the domain name using the master identity as the key has two nice properties, the public key will be different per-user per-website, solving the problem of linking user accounts cross-domain and re-use of passwords. And because the server is given the public key, it means that the website has no secrets to keep, significantly reducing the impact that a database breach might cause.
