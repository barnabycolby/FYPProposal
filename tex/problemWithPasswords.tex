Despite their well known flaws, usernames and passwords have been the standard system for authentication for many years, with all alternatives failing to succeed in replacing them. One of the core issues with passwords is the conflict that exists between two of the properties that they require. Firstly, a password must be strong, containing as much randomness as possible, and secondly, they must be memorable. But a strong random password is completely unmemorable, resulting in users choosing weak passwords that they can remember. Even if users could choose a strong password and take the time to remember it, they would likely reuse this password across multiple sites, to avoid having to remember so many passwords. The problem with this reuse is that we simply cannot trust 3rd parties to protect our secrets. Recent breaches, including Adobe, Snapchat and Patreon have proven this to be the case. Furthermore, email addresses are commonly used to identify users, providing the ability to connect user accounts across the web, a serious breach of privacy.
