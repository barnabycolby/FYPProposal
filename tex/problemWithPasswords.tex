Despite their well known flaws, usernames and passwords have been the standard system used for web-based authentication for many years, with all alternatives failing to succeed at replacing them. One of the core issues with passwords is the conflict that exists between two of the properties that they require. Firstly, a password must be strong, containing as much randomness as possible, and secondly, they must be memorable. But a strong random password is completely unmemorable, resulting in users choosing weak passwords that they can remember. This flaw is exacerbated by the sheer number of sites that require a password, it is simply not feasible to remember a strong password for every single one of them. Even if we manage to overcome these problems, our passwords are still at risk. History shows us that 3rd parties cannot be trusted to secure them.
